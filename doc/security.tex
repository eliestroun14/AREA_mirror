\documentclass[conference]{IEEEtran}

% --- Useful Packages ---
\usepackage[utf8]{inputenc}
\usepackage{graphicx}
\usepackage{hyperref}
\usepackage{amsmath, amssymb}
\usepackage{listings}
\usepackage{xcolor}

% --- Listings Configuration ---
\lstdefinelanguage{JavaScript}{
  keywords={typeof, new, true, false, catch, function, return, null, catch, switch, var, if, in, while, do, else, case, break, const, let, async, await, import, export, from, class, extends, constructor},
  keywordstyle=\color{blue}\bfseries,
  ndkeywords={class, export, boolean, throw, implements, import, this},
  ndkeywordstyle=\color{darkgray}\bfseries,
  identifierstyle=\color{black},
  sensitive=false,
  comment=[l]{//},
  morecomment=[s]{/*}{*/},
  commentstyle=\color{purple}\ttfamily,
  stringstyle=\color{red}\ttfamily,
  morestring=[b]',
  morestring=[b]"
}

\lstdefinelanguage{TypeScript}{
  keywords={typeof, new, true, false, catch, function, return, null, catch, switch, var, if, in, while, do, else, case, break, const, let, async, await, import, export, from, class, extends, constructor, interface, type, public, private, protected, readonly},
  keywordstyle=\color{blue}\bfseries,
  ndkeywords={class, export, boolean, throw, implements, import, this, string, number, boolean, void, any},
  ndkeywordstyle=\color{darkgray}\bfseries,
  identifierstyle=\color{black},
  sensitive=false,
  comment=[l]{//},
  morecomment=[s]{/*}{*/},
  commentstyle=\color{purple}\ttfamily,
  stringstyle=\color{red}\ttfamily,
  morestring=[b]',
  morestring=[b]"
}

\lstset{
  backgroundcolor=\color{white},
  basicstyle=\footnotesize\ttfamily,
  breakatwhitespace=false,
  breaklines=true,
  captionpos=b,
  deletekeywords={},
  escapeinside={\%*}{*},
  extendedchars=true,
  frame=none,
  keepspaces=true,
  language=JavaScript,
  morekeywords={*,...},
  numbers=left,
  numbersep=5pt,
  numberstyle=\tiny\color{gray},
  showspaces=false,
  showstringspaces=false,
  showtabs=false,
  stepnumber=1,
  stringstyle=\color{red},
  tabsize=2,
  title=\lstname,
  aboveskip=1em,
  belowskip=1em
}

% --- Paper Information ---
\title{Analysis of Security Vulnerabilities in a Modern Stack Based on React and NextJs, React Native, NestJS, and Prisma}

\author{
    \IEEEauthorblockN{STROUN Elie}
    \IEEEauthorblockA{Epitech Rennes\\
    elie.stroun@epitech.eu}

    \newline
    \\
    \IEEEauthorblockA{Contributors: Nathan Jeannot, Manech Dubreil, Aymeric Jouannet-Mimy, Pablo Jesus}
}

\thanks{The author would like to acknowledge the valuable contributions of Nathan Jeannot, Manech Dubreil, Aymeric Jouannet-Mimy, and Pablo Jesus to this paper.}

\begin{document}

\maketitle

\begin{abstract}
This paper presents a security study of a modern stack composed of React + NextJs (frontend), React Native (mobile), NestJS (backend), Prisma (ORM), and PostgreSQL (database). The objective is to identify known vulnerabilities, analyze their potential impact, and propose mitigation recommendations. This study is based on a literature review (CVE, OWASP) as well as simple experimental tests.
\end{abstract}

\begin{IEEEkeywords}
Security, React, React Native, NestJS, Prisma, PostgreSQL, OWASP, Vulnerabilities
\end{IEEEkeywords}

\section{Introduction}
Modern JavaScript/TypeScript frameworks such as React and NestJS have become standards in web and mobile development. However, their widespread adoption introduces new attack surfaces. This paper aims to examine the security flaws associated with this stack. We focus on the following technologies:
\begin{itemize}
    \item \textbf{Frontend:} React + NextJs
    \item \textbf{Mobile:} React Native
    \item \textbf{Backend:} NestJS
    \item \textbf{Database:} Prisma + PostgreSQL
\end{itemize}
We will analyze common vulnerabilities, their implications, and mitigation strategies.

\section{Related Work}
Briefly present the OWASP Top 10, studies on the security of JS frameworks, recent vulnerability reports (CVE).

The OWASP Top 10 is a regularly updated report outlining the most critical security risks to web applications. 
The 2021 edition highlights risks such as Injection, Broken Authentication, and Cross-Site Scripting (XSS) \cite{owasp2021}. 
The National Vulnerability Database (NVD) provides a comprehensive list of known vulnerabilities, including those affecting JavaScript frameworks \cite{nvd}. For instance, several CVEs have been reported for NestJS, including issues related to improper input validation and authentication bypass \cite{cveNest}. React, being one of the most popular frontend libraries, has also faced scrutiny. Common vulnerabilities include XSS attacks due to improper handling of user input and issues with component lifecycle methods that can lead to data leaks \cite{reactSecurity, reactFlaw}. NextJs, a modern build tool for frontend development, has had vulnerabilities such as arbitrary file read issues \cite{NextJsFlaw}.
For mobile applications, the OWASP Mobile Top 10 identifies risks like Insecure Data Storage and Insufficient Cryptography \cite{owaspMobile}. Numerous studies have examined the security of JavaScript frameworks. Many articles provide insights into common vulnerabilities and best practices for secure coding in these environments \cite{reactNativeSec}.

\section{Methodology}
\begin{itemize}
    \item Selection of technologies.
    \item Literature review and research in vulnerability databases.
\end{itemize}

\section{Security Analysis}
\subsection{React + NextJs (Frontend)}
React applications are susceptible to several security vulnerabilities, primarily due to improper handling of user input and state management. Common issues include Cross-Site Scripting (XSS), where malicious scripts are injected into the application, and Component Injection, where attackers exploit component properties to manipulate the application behavior \cite{reactSecurity, reactFlaw}.

\textbf{Vulnerable Code Example:}

\begin{lstlisting}[language=JavaScript, caption=Vulnerable React component susceptible to XSS]
// Vulnerable: Direct HTML injection
function UserProfile({ userBio }) {
  return (
    <div>
      <h2>User Bio</h2>
      <div dangerouslySetInnerHTML={{__html: userBio}} />
    </div>
  );
}

// Attacker input: userBio = "<script>alert('XSS')</script>"
\end{lstlisting}

\newpage

\textbf{Secure Code Example:}

\begin{lstlisting}[language=JavaScript, caption=Secure React component with XSS protection]
import DOMPurify from 'dompurify';

function UserProfile({ userBio }) {
  const sanitizedBio = DOMPurify.sanitize(userBio);
  
  return (
    <div>
      <h2>User Bio</h2>
      <div dangerouslySetInnerHTML={{__html: sanitizedBio}} />
    </div>
  );
}
\end{lstlisting}

NextJs, as a build tool, can also introduce vulnerabilities if not configured properly. Issues such as arbitrary file reads and exposure of sensitive information through source maps have been reported \cite{NextJsFlaw}.

\textbf{NextJS Configuration Vulnerability:}

\begin{lstlisting}[language=JavaScript, caption=Insecure NextJS configuration]
// next.config.js - Vulnerable configuration
module.exports = {
  productionBrowserSourceMaps: true, // Exposes source code
  publicRuntimeConfig: {
    API_SECRET: process.env.API_SECRET // Exposed to client
  }
}
\end{lstlisting}

\textbf{Secure NextJS Configuration:}

\begin{lstlisting}[language=JavaScript, caption=Secure NextJS configuration]
// next.config.js - Secure configuration
module.exports = {
  productionBrowserSourceMaps: false, // Hide source code
  serverRuntimeConfig: {
    API_SECRET: process.env.API_SECRET // Server-side only
  },
  publicRuntimeConfig: {
    API_URL: process.env.PUBLIC_API_URL // Only public data
  },
  headers: async () => {
    return [
      {
        source: '/(.*)',
        headers: [
          {
            key: 'X-Content-Type-Options',
            value: 'nosniff'
          },
          {
            key: 'X-Frame-Options',
            value: 'DENY'
          }
        ]
      }
    ];
  }
}
\end{lstlisting}

\subsection{React Native (Mobile)}
React Native applications face unique security challenges, particularly in the areas of data storage and communication. The OWASP Mobile Top 10 highlights risks such as Insecure Data Storage, where sensitive information is not adequately protected on the device \cite{owaspMobile}.

\textbf{Vulnerable Data Storage:}

\begin{lstlisting}[language=JavaScript, caption=Insecure data storage in React Native]
import AsyncStorage from '@react-native-async-storage/async-storage';

// Vulnerable: Storing sensitive data in plain text
const storeUserCredentials = async (username, password) => {
  await AsyncStorage.setItem('username', username);
  await AsyncStorage.setItem('password', password); // Plain text!
};
\end{lstlisting}

\textbf{Secure Data Storage:}

\begin{lstlisting}[language=JavaScript, caption=Secure data storage using Keychain]
import * as Keychain from 'react-native-keychain';

const storeUserCredentials = async (username, password) => {
  await Keychain.setCredentials(username, password, {
    accessControl: Keychain.ACCESS_CONTROL.BIOMETRY_CURRENT_SET,
    authenticationType: Keychain.AUTHENTICATION_TYPE.BIOMETRICS,
  });
};
\end{lstlisting}

Additionally, React Native apps should implement proper SSL pinning to prevent man-in-the-middle attacks:

\textbf{Network Security Implementation:}

\begin{lstlisting}[language=JavaScript, caption=Secure network communication with SSL pinning]
import axios from 'axios';
import { NetworkingModule } from 'react-native';

// Configure SSL pinning
const secureApiClient = axios.create({
  baseURL: 'https://api.example.com',
  timeout: 10000,
  headers: {
    'X-API-Key': 'your-api-key',
    'Content-Type': 'application/json'
  }
});

// Add certificate pinning validation
secureApiClient.interceptors.request.use((config) => {
  // Validate certificate fingerprint
  return config;
});
\end{lstlisting}

\subsection{NestJS (Backend)}
NestJS, being a backend framework, is susceptible to different types of security vulnerabilities compared to frontend frameworks. Common issues include improper authentication and authorization mechanisms \cite{nestjs}.

\textbf{Vulnerable Authentication:}

\begin{lstlisting}[language=TypeScript, caption=Vulnerable NestJS endpoint without proper validation]
@Controller('users')
export class UsersController {
  @Get(':id')
  async getUser(@Param('id') id: string) {
    // Vulnerable: No authentication or authorization check
    return this.usersService.findById(id);
  }

  @Post('update')
  async updateUser(@Body() userData: any) {
    // Vulnerable: No input validation
    return this.usersService.update(userData);
  }
}
\end{lstlisting}

\textbf{Secure Implementation:}

\begin{lstlisting}[language=TypeScript, caption=Secure NestJS endpoint with proper guards and validation]
import { UseGuards, UsePipes, ValidationPipe } from '@nestjs/common';
import { JwtAuthGuard } from './auth/jwt-auth.guard';
import { IsString, IsEmail } from 'class-validator';

class UpdateUserDto {
  @IsString()
  name: string;
  
  @IsEmail()
  email: string;
}

@Controller('users')
@UseGuards(JwtAuthGuard)
export class UsersController {
  @Get(':id')
  async getUser(@Param('id') id: string, @Req() req) {
    // Check if user can access this resource
    if (req.user.id !== id && !req.user.isAdmin) {
      throw new ForbiddenException();
    }
    return this.usersService.findById(id);
  }

  @Post('update')
  @UsePipes(new ValidationPipe())
  async updateUser(@Body() userData: UpdateUserDto) {
    return this.usersService.update(userData);
  }
}
\end{lstlisting}

\subsection{Prisma + PostgreSQL}
Prisma, as an ORM, can help mitigate SQL injection risks by using parameterized queries. However, developers must still be cautious about how they construct queries \cite{databaseFlaw}.

\textbf{Vulnerable Query Construction:}

\begin{lstlisting}[language=TypeScript, caption=Vulnerable raw SQL query in Prisma]
// Vulnerable: Raw SQL with string concatenation
async searchUsers(searchTerm: string) {
  return this.prisma.$queryRaw`
    SELECT * FROM users 
    WHERE name LIKE '%${searchTerm}%'
  `; // SQL injection possible!
}
\end{lstlisting}

\textbf{Secure Query Implementation:}

\begin{lstlisting}[language=TypeScript, caption=Secure Prisma query with proper parameterization]
// Secure: Using Prisma's type-safe queries
async searchUsers(searchTerm: string) {
  return this.prisma.user.findMany({
    where: {
      name: {
        contains: searchTerm,
        mode: 'insensitive'
      }
    }
  });
}

// If raw SQL is necessary, use proper parameterization
async searchUsersRaw(searchTerm: string) {
  return this.prisma.$queryRaw`
    SELECT * FROM users 
    WHERE name ILIKE ${`%${searchTerm}%`}
  `;
}
\end{lstlisting}

\section{Discussion}
The analysis reveals that while each component of the stack has its own set of vulnerabilities, many of these can be mitigated through best practices in secure coding, configuration, and regular updates. For instance, using libraries like DOMPurify can help prevent XSS attacks in React applications, while implementing strong authentication and authorization mechanisms can secure NestJS backends \cite{nestjsSec}. 

\textbf{Comprehensive Security Middleware Example:}

\begin{lstlisting}[language=TypeScript, caption=Security middleware implementation for NestJS]
import helmet from 'helmet';
import rateLimit from 'express-rate-limit';

// Security configuration
app.use(helmet({
  contentSecurityPolicy: {
    directives: {
      defaultSrc: ["'self'"],
      scriptSrc: ["'self'", "'unsafe-inline'"],
      styleSrc: ["'self'", "'unsafe-inline'"],
    },
  },
}));

// Rate limiting
app.use(rateLimit({
  windowMs: 15 * 60 * 1000, // 15 minutes
  max: 100, // limit each IP to 100 requests per windowMs
  message: 'Too many requests from this IP'
}));
\end{lstlisting}

Regularly updating dependencies and monitoring vulnerability databases like NVD for new CVEs is also crucial for maintaining security across the stack.
Ideally, a comprehensive security strategy should take into account the specific risks associated with each technology and implement layered defenses to protect against a wide range of threats.

\section{Conclusion and Perspectives}
This study highlights the importance of understanding and addressing security vulnerabilities in modern web and mobile development stacks with a growing number of users and threats. By adopting best practices and staying informed about emerging threats, developers can significantly reduce the risk of security breaches. Real threats exists and are most of the time the ones that we don't know yet. Future work could involve developing automated tools for vulnerability detection and mitigation specific to this stack.


% References
\begin{thebibliography}{00}
\bibitem{owasp} OWASP, \emph{OWASP Top 10 Web and Mobile}, 2021.
\bibitem{cve} National Vulnerability Database, \url{https://nvd.nist.gov/}.
\bibitem{nestjs} NestJS Documentation, \url{https://nestjs.com/}.
\bibitem{react} React Documentation, \url{https://react.dev/}.
\bibitem{owasp2021} OWASP Foundation, ``OWASP Top Ten Web Application Security Risks 2021 \url{https://owasp.org/Top10/}, 2021.
\bibitem{owaspMobile} OWASP Foundation, ``OWASP Mobile Top Ten Security Risks 2016 \url{https://owasp.org/www-project-mobile-top-10/}, 2016.
\bibitem{nvd} National Institute of Standards and Technology (NIST), ``National Vulnerability Database (NVD) \url{https://nvd.nist.gov/}, 2025.
\bibitem{cveNest} MITRE, ``Common Vulnerabilities and Exposures (CVE) for NestJS \url{https://cve.mitre.org/}, 2025.

\bibitem{reactSecurity} Z. Guo, M. Kang, V.N. Venkatakrishnan, R. Gjomemo and Y.Cao ``ReactAppScan: Mining React Application Vulnerabilities via Component Graph \url{https://dl.acm.org/doi/10.1145/3658644.3670331}, 2024.
\bibitem{reactFlaw} Scofield O. Idehen, ``Top Security Vulnerabilities in React Applications \url{https://medium.com/@Scofield_Idehen/top-security-vulnerabilities-in-react-applications-b67f1c375494}, 2023.
\bibitem{NextJsFlaw} National Cyber Security Centre, ``Vulnerability affecting Next.js web development framework CVE-2025-29927 \\ \url{https://www.ncsc.gov.uk/news/vulnerability-affecting-nextjs-web-development-framework}, 2025.
\bibitem{databaseFlaw} Floris Van den Abeele, ``Prisma and PostgreSQL vulnerable to NoSQL injection? A surprising security risk explained \url{https://fr.aikido.dev/blog/prisma-and-postgresql-vulnerable-to-nosql-injection}, 2025.
\bibitem{reactNativeFlaw} Ddos, ``Critical RCE Flaw (CVE-2025-54594) in React Native Bottom Tabs’ GitHub Actions Exposed Secrets \\ \url{https://securityonline.info/critical-rce-flaw-cve-2025-54594-in-react-native-bottom-tabs-github-actions-exposed-secrets/}, 2025.
\bibitem{reactNativeSec} Ankit Dhawan on Medium, ``Secure Your React Native App from Vulnerabilities \url{https://medium.com/@ankit.ad.dhawan/secure-your-react-native-app-from-vulnerabilities-e23b5cb00b2a}, 2024.
\bibitem{nestjsFlaw} Brittany Day on Linux Security,  ``Critical NestJS Vulnerability Exposes Developers to RCE Risk \url{https://linuxsecurity.com/news/security-vulnerabilities/critical-nestjs-rce-vulnerability}, 2025.
\bibitem{nestjsSec} Ankit on Medium, ``How to Prevent Security Vulnerabilities in a NestJS API Before Deployment \\ \url{https://medium.com/@mohantaankit2002/how-to-prevent-security-vulnerabilities-in-a-nestjs-api-before-deployment-3c2eaa05e0ca}, 2025.

\end{thebibliography}

\end{document}